\documentclass{beamer}

\mode<presentation> {

%%%%%%%THEME%%CHOICE%%%%%%

%\usetheme{default}
%\usetheme{AnnArbor}
%\usetheme{Antibes}
%\usetheme{Bergen}
%\usetheme{Berkeley}
%\usetheme{Berlin}
%\usetheme{Boadilla}
%\usetheme{CambridgeUS}
%\usetheme{Copenhagen}
%\usetheme{Darmstadt}
%\usetheme{Dresden}
%\usetheme{Frankfurt}
%\usetheme{Goettingen}
%\usetheme{Hannover}
%\usetheme{Ilmenau}
%\usetheme{JuanLesPins}
%\usetheme{Luebeck}
\usetheme{Madrid}
%\usetheme{Malmoe}
%\usetheme{Marburg}
%\usetheme{Montpellier}
%\usetheme{PaloAlto}
%\usetheme{Pittsburgh}
%\usetheme{Rochester}
%\usetheme{Singapore}
%\usetheme{Szeged}
%\usetheme{Warsaw}

%%%%%COLOUR%%PALLETTE%%%%%

%\usecolortheme{albatross}
\usecolortheme{beaver}
%\usecolortheme{beetle}
%\usecolortheme{crane}
%\usecolortheme{dolphin}
%\usecolortheme{dove}
%\usecolortheme{fly}
%\usecolortheme{lily}
%\usecolortheme{orchid}
%\usecolortheme{rose}
%\usecolortheme{seagull}
%\usecolortheme{seahorse}
%\usecolortheme{whale}
%\usecolortheme{wolverine}

%%%%%REMOVE%%FOOTER%%%%%

%\setbeamertemplate{footline} 

%%%%%REPLACE%%FOOTER%%WITH%%SLIDE%%COUNT%%%%%

%\setbeamertemplate{footline}[page number] 

%%%%%REMOVE%%NAVI%%SYMBOLS%%%%%

%\setbeamertemplate{navigation symbols}{} 
}

\usepackage{graphicx} 
\usepackage{booktabs} 
\usepackage{multicol}
\usepackage{tikz}
\usepackage{bussproofs}
\setbeamercolor{block title}{bg=red!30,fg=black}
\setbeamertemplate{itemize item}{\color{red!35}$\blacksquare$}
\setbeamercolor{local structure}{fg=darkred}
%show TOCs highlighting current section at beginning of section
\AtBeginSection[]
{
    \begin{frame}
        \frametitle{Table of Contents}
        \tableofcontents[currentsection]
    \end{frame}
}
\newcommand{\du}{\Diamond_\uparrow}
\newcommand{\dl}{\Diamond_\leftarrow}
\newcommand{\bu}{\Box_\uparrow}
\newcommand{\bl}{\Box_\leftarrow}
\title[Axiomatic Potentialism]{Axiomatic Potentialism} 
% The short title appears at the bottom of every slide, 
% the full title is only on the title page

\author{Chris Scambler} 
\institute[ASC] 
% Your institution as it will appear on the bottom of every slide, 
% may be shorthand to save space
{
All Souls College, \\
Oxford University \\ 
\medskip
\textit{chris.scambler@all-souls.ox.ac.uk} 
}
\date{\today} 

\begin{document}

\begin{frame}
\titlepage 
\end{frame}

\begin{frame}
\frametitle{Overview} 
\tableofcontents 
\end{frame}
\section{Background}

\begin{frame}
\frametitle{Potentialism}
\onslide<2->
\begin{block}
    {\bf Potentialism} is the idea that a mathematical object (e.g. a set) is the sort 
    of thing that may \emph{merely possibly} exist.
\end{block}
\begin{itemize}
    \item<3-> E.g. a geometric object as a figure one can construct
    \item<4-> A set as a certain sort of data structure one could assemble
    \item<5-> Or perhaps a structure that is instantiated given enough objects.
    \item<6-> The idea has deep roots in set theory, e.g. Zermelo and even Cantor
    \item<7-> Still deeper roots in mathematics in general.
\end{itemize}
\end{frame}

\begin{frame}
\frametitle{Potentialism}
    \begin{itemize}
        \item The recent literature has seen two branches of study here:
        \begin{enumerate}
            \item<2->   Model-theoretic: study Kripke models whose worlds are 
                        structures with the accessibility relation (some refinement of)
                        the substructure relation.
            \item<3->   Axiomatic: Develop axiom systems designed to characterize 
                        this or that form of potentialism directly, without appeal to models.
        \end{enumerate}
        \item<4->   In each case interesting questions arise concerning the relation 
                    between assertions in the modal framework and in first order set theory.
        \item<5->   E.g. Hamkins and Linnebo showed in MT potentialism with the structures
                    initial segments of $V$ that $S5$ at a world $V_\kappa$ is equivalent 
                    to $\Sigma_3$ correctness of $\kappa$.
        \item<6->   Here we will be focussed on axiomatic potentialism, and on 
                    relations between potentialist axiom systems and their first order counterparts.
    \end{itemize}
    
\end{frame}

\section{Warm Up: Height Potentialism}

\begin{frame}
    \frametitle{Motivation}
    \begin{itemize}
        \item<2->   Imagine one has the ability to take things and 
                    make a set containing them.
        \item<3->   Imagine one is able to do this arbitrarily many times.
        \item<4->   Axiomatize this conception and relate it to standard set theory.
    \end{itemize}
\end{frame}

\begin{frame}
    \frametitle{Language}
    \begin{block}{The Language $\mathcal{L}_0$}
        \begin{itemize}
            \item object variables $x, y, z$
            \item<2-> plural variables $X, Y, Z$
            \item<3-> $\wedge, \neg, \forall, =$
            \item<4-> $\Box$
            \item<5-> $\in$
        \end{itemize} 
    \end{block}
\end{frame}

\begin{frame}
    \frametitle{Axioms for the theory $\mathsf{L}$}
    \begin{block}{Logical Axioms}
    \begin{enumerate}
        \item<2-> Free FO logic
        \item<3-> $\mathsf{S4.2}$ modal logic + CBF
        \item<4-> Ext for $X$, $\Diamond Xx \rightarrow \Box Xx$, 
        $\Diamond \exists x[Xx \wedge x = y] \rightarrow \exists x[Xx \wedge x = y]$,
        Choice, Comp
    \end{enumerate}
    \end{block}
    \onslide<5->
    \begin{block}{Set-theoretic axioms}
        \begin{enumerate}
        \item<6-> Extensionality, $\in$-rigidity, foundation
        \item<7-> $\Box \forall X \Diamond \exists x [Set(x, X)]$
        \item<9-> $\Diamond \exists X \Box \forall x[Xx \leftrightarrow \mathbb{N}x]$
        \item<10-> $\Diamond \exists X \Box \forall x[Xx \leftrightarrow x \subseteq y]$
        \item<11-> A modal translation of replacement
        \end{enumerate}
    \end{block}
    \onslide<8-> NB: $\Box \exists X \neg \exists y [Set(x, X)]$
\end{frame}
\begin{frame}
    \frametitle{Inconsistency?}
    Standard modal model theory validates the rule
    \begin{prooftree}
        \AxiomC{$\varphi \rightarrow \Box \psi$}
        \UnaryInfC{$\varphi \rightarrow \Box \forall x \psi$}
    \end{prooftree}
\onslide<2->
    \begin{equation}
        (Xx \leftrightarrow x \not \in x) \rightarrow \forall y \Box \neg Set(y, X)
    \end{equation}
\onslide<3->
    \begin{equation}
        (Xx \leftrightarrow x \not \in x) \rightarrow \Box \neg Set(y, X)
    \end{equation}
\onslide<4->
    \begin{equation}
        (Xx \leftrightarrow x \not \in x) \rightarrow \Box \forall y \neg Set(y, X)
    \end{equation}
\onslide<5->
{\bf Hence the need for free logic.}
\end{frame}
\begin{frame}
    \frametitle{Relative Consistency}
\begin{itemize}
    \item In fact ZFC interprets $\mathsf{L}$.
\end{itemize}
\onslide<2->
\begin{block}{$t : \mathcal{L}_0 \times V \to \mathcal{L}_\in, (\varphi, T) \mapsto \psi(T)$}
    \begin{itemize}
        \item<3-> assign plural variables odd numbered variables $t(X)$.
        \item<4-> Membership claims on sets, propositional connectives = id 
        \item<5-> $t(Xx)(T) := x \in t(X)$
        \item<6-> $t(\forall x \varphi)(T) := \forall x \in T [t(\varphi)(T)]$
        \item<7-> $t(\forall X \varphi)(T) := \forall x \subseteq T [t(\varphi)(T)]$
        \item<8-> $t(\Box \varphi)(t) := \forall S \supseteq T [Tran (S) \rightarrow t(\varphi)(S)]$
    \end{itemize}
\end{block}
\onslide<9->
\begin{block}{Theorem}
  $\mathsf{L} \vdash \varphi$ implies $ZFC \vdash t(\varphi)(\emptyset)$  
\end{block}
\end{frame}
\begin{frame}
    \frametitle{Converse Translation}
\onslide<2->
\begin{block}{Mirroring theorem}
\onslide<3-> For $\varphi$ in $\mathcal{L}_\in$, let $\varphi^\diamond$ be the result of 
prefixing all universal quantifiers by a $\Box$ 
(and existential quantifiers by $\Diamond$.) Then we have 
\onslide<4->
\[
    \Gamma \vdash_{FOL} \varphi 
    \Leftrightarrow 
    \Gamma^\diamond \vdash_{\mathsf{L}} \varphi^\diamond
\]
\end{block}
\onslide<5-> Note on replacement$^\diamond$.
\onslide<6->
\begin{block}{Linnebo Interpretation Theorem}
  $\mathsf{L} \vdash ZFC^\diamond$.  
\end{block}
\onslide<7-> Proof: use mirroring.
\end{frame}
\begin{frame}
    \frametitle{Axioms for the theory $\mathsf{L}$}
    \begin{block}{Logical Axioms}
    \begin{enumerate}
        \item Free FO logic
        \item $\mathsf{S4.2}$ modal logic + CBF
        \item Ext for $X$, $\Diamond Xx \rightarrow \Box Xx$, 
        $\Diamond \exists x[Xx \wedge x = y] \rightarrow \exists x[Xx \wedge x = y]$,
        Choice, Comp
    \end{enumerate}
    \end{block}
    \begin{block}{Set-theoretic axioms}
        \begin{enumerate}
        \item Extensionality, $\in$-rigidity, foundation
        \item $\Box \forall X \Diamond \exists x [Set(x, X)]$
        \item $\Diamond \exists X \Box \forall x[Xx \leftrightarrow \mathbb{N}x]$
        \item $\Diamond \exists X \Box \forall x[Xx \leftrightarrow x \subseteq y]$
        \item A modal translation of replacement
        \end{enumerate}
    \end{block}
\end{frame}

\begin{frame}
    \frametitle{Mini-conclusion}
\onslide<2->
\begin{block}{Equivalence}
    We have an exact proof-theoretic equivalence, $\mathsf{L} \equiv ZFC$.
\end{block}
\end{frame}

\section{Height and Width Potentialism}
\begin{frame}
    \frametitle{Motivation}
    \begin{itemize}
        \item<2->   Imagine one has the ability to take things and 
                    make a set containing them.
        \item<3->   Imagine one is also able to take a partial order and add a filter 
                    meeting all its (current) dense sets; 
        \item<4->   Or, equivalently, to take some things and add an enumerating function.
        \item<5->   Axiomatize this conception and relate it to standard set theory.
    \end{itemize}
\end{frame}
\begin{frame}
    \frametitle{Language}
    \begin{block}{The Language $\mathcal{L}_1$}
        \begin{itemize}
            \item object variables $x, y, z$
            \item<2-> plural variables $X, Y, Z$
            \item<3-> $\wedge, \neg, \forall, =$
            \item<4-> $\bu$, $\bl$, $\Box$
            \item<5-> $\in$
        \end{itemize} 
    \end{block}
\end{frame}
\begin{frame}
    \frametitle{Axioms for the theory $\mathsf{M}$}
    \begin{block}{Logical Axioms}
    \begin{enumerate}
        \item<2-> Free FO logic
        \item<3-> $\mathsf{S4.2}$ modal logic + CBF for each modal
        \item<4-> $\Box \varphi \rightarrow \bu \varphi$, same for $\bl$.
        \item<4-> Ext for $X$, $\Diamond Xx \rightarrow \Box Xx$, 
        $\Diamond \exists x[Xx \wedge x = y] \rightarrow \exists x[Xx \wedge x = y]$,
        Choice, Comp
    \end{enumerate}
    \end{block}
    \onslide<5->
    \begin{block}{Set-theoretic axioms}
        \begin{enumerate}
        \item<6-> Extensionality, $\in$-rigidity, foundation
        \item<7-> $\Box \forall X \du \exists x [Set(x, X)]$
        \item<9-> $\du \exists X \Box \forall x[Xx \leftrightarrow \mathbb{N}x]$
        \item<10-> $\du \exists X \bu \forall x[Xx \leftrightarrow x \subseteq y]$
        \item<11-> $\Box \forall x, X [D(x, X)\rightarrow \dl \exists g[Fmeets(g, xx)]]$
        \item<12-> A modal translation of replacement
        \end{enumerate}
    \end{block}
\end{frame}
\begin{frame}{Some initial observations}
    \begin{itemize}
        \item<2-> $\mathsf{M}$ interprets ZFC under the translation 
                    $\varphi \mapsto \varphi^{\diamond_\uparrow}$. 
        \item<3-> $\mathsf{M}$ interprets ZFC$^-$ under the translation 
                    $\varphi \mapsto \varphi^\diamond$.
        \item<4-> $\mathsf{M}$ proves $\neg Pow^\diamond$.
        \item<5-> $\mathsf{M}$ proves $V = HC^\diamond$ and hence $SOA^\diamond$.
        \item<5-> $\mathsf{M}$ proves it is possible for the continuum 
                    to exist and have a cardinality at least as great as 
                    any $\aleph$ number whose existence is provable 
                    in ZFC.
    \end{itemize}
\end{frame}
\begin{frame}
    \frametitle{Inconsistency?}
The axioms imply 
\[
    \Diamond \exists x (\Diamond_\leftarrow \exists y[y \subseteq x \wedge y = z] \wedge \Box_\uparrow \neg \exists y[y = z])
\]
\onslide<2-> 
abbreviate the formula in parentheses by $\Psi(x, z)$. 

\onslide<3-> 
By comprehension,
\[
    \Diamond \exists x \Psi(x, z) \wedge \exists X \forall y[Xy \leftrightarrow y \in z]
\]

\onslide<4-> 
By height potentialism/rigidty,
\[
    \Diamond \exists x \Psi(x, z) \wedge \du \exists w \forall y[y \in w \leftrightarrow y \in z]
\]
\onslide<5-> But then the rigidity/extensionality imply $w = z$ after all, 
so we have a contradiction.
\end{frame}
\begin{frame}
    \frametitle{Resolution}
The argument just sketched uses comprehension with arbitrary parameters:
\[\exists X\forall y[Xy \leftrightarrow y \in z]\]
\onslide<2-> And in the crucial application, it applies when we have no 
\emph{a priori} guarantee $z$ even exists (indeed this is what we are trying 
to establish.)
\onslide<3-> Natural solution: restrict comp to closed form:
\[
    \Box \forall z \Box \forall Z \exists X \forall y[Xy \leftrightarrow \varphi(y, z, Z)]
\]
\onslide<4-> Amounts to restricting ourselves to parameters that exist at 
the world of evaluation.
\end{frame}

\begin{frame}
    \frametitle{Relative Consistency}
\begin{block}{Intuitive idea}
    \onslide<2-> (From now on, I will ignore the difference between SOA 
    and ZFC$^-$+ V = HC. Replacement is formulated as collection.)
    \begin{itemize}
    \item<3-> We will use the fact that $\mathsf{T}$ = SOA + $\Pi_1^1$-PSP 
    $\equiv$ ZFC, and in fact $\mathsf{T}$ proves that $L[r]$ is a 
    model of ZFC for every real $r$. 
    \item<4-> 
    Our translation will be doubly parameterized, once by a real and 
    once by a transitive set.
    \item<5->
    Our interpretation for $\du$ will involve holding $r$ fixed and 
    climbing transitive sets in $L[r]$;
    \item<6-> 
    while our interpretation for $\dl$ will involve allowing new reals 
    to be added but not extending the height of the transitive set parameter.
    \end{itemize}
\end{block}
\end{frame}
\begin{frame}
    \frametitle{Relative consistency}
    \onslide<2-> Let $M \vDash SOA + \Pi_1^1 PSP$.
    \onslide<3->
    \begin{block}{$t : \mathcal{L}_1 \times M \times \mathbb{R}^M \to \mathcal{L}_\in, (\varphi, T, r) \mapsto \psi(T, r)$}
        \begin{itemize}
            \item<4-> assign plural variables odd numbered variables $t(X)$.
            \item<5-> Membership claims on sets, propositional connectives = id 
            \item<6-> $t(Xx)(T, r) := x \in t(X)$
            \item<7-> $t(\forall x \varphi)(T, r) := \forall x \in T [t(\varphi)(T, r)]$
            \item<8-> $t(\forall X \varphi)(T, r) := \forall x \subseteq T [x \in L[r] \rightarrow t(\varphi)(T, r)]$
            \item<9-> $t(\bu \varphi)(T, r) := \forall S \supseteq T [Tran(S) \wedge S \in L[r] \rightarrow t(\varphi)(S, r)]$
            \item<10-> $t(\bl \varphi)(T, r) := \forall s[ r \in L[s] \rightarrow \forall S[ S \leq T \wedge S \in L[s] \rightarrow t(\varphi)(S, s)]]$
        \end{itemize}
    \end{block}
    \onslide<11->
\begin{block}{Theorem}
  $\mathsf{M} \vdash \varphi$ implies $\mathsf{T} \vdash t(\varphi)(\emptyset, 0)$  
\end{block}
\end{frame}
\begin{frame}
    \frametitle{Converse Translation}
\onslide<2->
\begin{block}{Theorem}
$\mathsf{M} \vdash \Pi_1^1 PSP^\diamond$
\end{block}
\onslide<3->
\begin{block}{Proof (sketch)}
    \onslide<4-> Given any possible real $r$, one can show $\Diamond \exists x[ x = \mathbb{R}^{L[r]}]$
    using the $\Diamond_\uparrow$ translation of ZFC and some absoluteness lemmas. 
    \onslide<5-> One can also show by forcing that ``$\Diamond \mathbb{R}^{L[r]} \text{ is countable}$.''
    \onslide<6-> This yields the $\Diamond$-translation of ``$\mathbb{R}^{L[r]}$ exists for every $r$, and is coutnable.''
    \onslide<7-> By an old result of Solovay, if $A$ is $\Sigma^1_2$ definable in $r$, then 
    either $A \in L[r]$ or $A$ contains a perfect subset. Hence the above implies (in standard 
    non modal set theory) that the perfect set property for $\Sigma^1_2$ sets holds, i.e. $\Pi_1^1 PSP$.
    \onslide<8-> But now the mirroring theorem implies the corresponding modal result.
    
\end{block}
\end{frame}
\begin{frame}
\frametitle{References}
\footnotesize{
\begin{thebibliography}{99}
    \bibitem[Smith, 2012]{p1} John Smith (2012)
    \newblock Title of the publication
    \newblock \emph{Journal Name} 12(3), 45 -- 678.
    \end{thebibliography}
}
\end{frame}

\begin{frame}
\Huge{\centerline{Thanks!}}
\end{frame}

\end{document} 