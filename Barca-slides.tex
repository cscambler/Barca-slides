\documentclass{beamer}

\mode<presentation> {

%%%%%%%THEME%%CHOICE%%%%%%

%\usetheme{default}
%\usetheme{AnnArbor}
%\usetheme{Antibes}
%\usetheme{Bergen}
%\usetheme{Berkeley}
%\usetheme{Berlin}
%\usetheme{Boadilla}
%\usetheme{CambridgeUS}
%\usetheme{Copenhagen}
%\usetheme{Darmstadt}
%\usetheme{Dresden}
%\usetheme{Frankfurt}
%\usetheme{Goettingen}
%\usetheme{Hannover}
%\usetheme{Ilmenau}
%\usetheme{JuanLesPins}
%\usetheme{Luebeck}
\usetheme{Madrid}
%\usetheme{Malmoe}
%\usetheme{Marburg}
%\usetheme{Montpellier}
%\usetheme{PaloAlto}
%\usetheme{Pittsburgh}
%\usetheme{Rochester}
%\usetheme{Singapore}
%\usetheme{Szeged}
%\usetheme{Warsaw}

%%%%%COLOUR%%PALLETTE%%%%%

%\usecolortheme{albatross}
\usecolortheme{beaver}
%\usecolortheme{beetle}
%\usecolortheme{crane}
%\usecolortheme{dolphin}
%\usecolortheme{dove}
%\usecolortheme{fly}
%\usecolortheme{lily}
%\usecolortheme{orchid}
%\usecolortheme{rose}
%\usecolortheme{seagull}
%\usecolortheme{seahorse}
%\usecolortheme{whale}
%\usecolortheme{wolverine}

%%%%%REMOVE%%FOOTER%%%%%

%\setbeamertemplate{footline} 

%%%%%REPLACE%%FOOTER%%WITH%%SLIDE%%COUNT%%%%%

%\setbeamertemplate{footline}[page number] 

%%%%%REMOVE%%NAVI%%SYMBOLS%%%%%

%\setbeamertemplate{navigation symbols}{} 
}

\usepackage{graphicx} 
\usepackage{booktabs} 
\usepackage{multicol}
\usepackage{tikz}
\setbeamercolor{block title}{bg=red!30,fg=black}
\setbeamertemplate{itemize item}{\color{red!35}$\blacksquare$}
\setbeamercolor{local structure}{fg=darkred}
%show TOCs highlighting current section at beginning of section
\AtBeginSection[]
{
    \begin{frame}
        \frametitle{Table of Contents}
        \tableofcontents[currentsection]
    \end{frame}
}

\title[Axiomatic Potentialism]{Axiomatic Potentialism} 
% The short title appears at the bottom of every slide, 
% the full title is only on the title page

\author{Chris Scambler} 
\institute[ASC] 
% Your institution as it will appear on the bottom of every slide, 
% may be shorthand to save space
{
All Souls College, \\
Oxford University \\ 
\medskip
\textit{chris.scambler@all-souls.ox.ac.uk} 
}
\date{\today} 

\begin{document}

\begin{frame}
\titlepage 
\end{frame}

\begin{frame}
\frametitle{Overview} 
\tableofcontents 
\end{frame}
\section{Background}

\begin{frame}
\frametitle{Potentialism}
\onslide<2->
\begin{block}
    {\bf Potentialism} is the idea that a mathematical object (e.g. a set) is the sort 
    of thing that may \emph{merely possibly} exist.
\end{block}
\begin{itemize}
    \item<3-> E.g. a geometric object as a figure one can construct
    \item<4-> A set as a certain sort of data structure one could assemble
    \item<5-> Or perhaps a structure that is instantiated given enough objects.
    \item<6-> The idea has deep roots in set theory, e.g. Zermelo and even Cantor
    \item<7-> Still deeper roots in mathematics in general.
\end{itemize}
\end{frame}

\begin{frame}
\frametitle{Potentialism}
    \begin{itemize}
        \item The recent literature has seen two branches of study here:
        \begin{enumerate}
            \item<2->   Model-theoretic: study Kripke models whose worlds are 
                        structures with the accessibility relation (some refinement of)
                        the substructure relation.
            \item<3->   Axiomatic: Develop axiom systems designed to characterize 
                        this or that form of potentialism directly, without appeal to models.
        \end{enumerate}
        \item<4->   In each case interesting questions arise concerning the relation 
                    between assertions in the modal framework and in first order set theory.
        \item<5->   E.g. Hamkins and Linnebo showed in MT potentialism with the structures
                    initial segments of $V$ that $S5$ at a world $V_\kappa$ is equivalent 
                    to $\Sigma_3$ correctness of $\kappa$.
        \item<6->   Here we will be focussed on axiomatic potentialism, and on 
                    relations between potentialist axiom systems and their first order counterparts.
    \end{itemize}
    
\end{frame}

\section{Warm Up: Height Potentialism}

\begin{frame}
    \frametitle{Motivation}
    Sed iaculis dapibus gravida. Morbi sed tortor erat, nec interdum arcu. Sed id lorem lectus. Quisque viverra augue id sem ornare non aliquam nibh tristique. Aenean in ligula nisl. Nulla sed tellus ipsum. Donec vestibulum ligula non lorem vulputate fermentum accumsan neque mollis.\\~\\
    
    Sed diam enim, sagittis nec condimentum sit amet, ullamcorper sit amet libero. Aliquam vel dui orci, a porta odio. Nullam id suscipit ipsum. Aenean lobortis commodo sem, ut commodo leo gravida vitae. Pellentesque vehicula ante iaculis arcu pretium rutrum eget sit amet purus. Integer ornare nulla quis neque ultrices lobortis. Vestibulum ultrices tincidunt libero, quis commodo erat ullamcorper id.
\end{frame}
\begin{frame}
    \frametitle{Axioms}
    Sed iaculis dapibus gravida. Morbi sed tortor erat, nec interdum arcu. Sed id lorem lectus. Quisque viverra augue id sem ornare non aliquam nibh tristique. Aenean in ligula nisl. Nulla sed tellus ipsum. Donec vestibulum ligula non lorem vulputate fermentum accumsan neque mollis.\\~\\
    
    Sed diam enim, sagittis nec condimentum sit amet, ullamcorper sit amet libero. Aliquam vel dui orci, a porta odio. Nullam id suscipit ipsum. Aenean lobortis commodo sem, ut commodo leo gravida vitae. Pellentesque vehicula ante iaculis arcu pretium rutrum eget sit amet purus. Integer ornare nulla quis neque ultrices lobortis. Vestibulum ultrices tincidunt libero, quis commodo erat ullamcorper id.
\end{frame}
\begin{frame}
    \frametitle{Inconsistency?}
    Sed iaculis dapibus gravida. Morbi sed tortor erat, nec interdum arcu. Sed id lorem lectus. Quisque viverra augue id sem ornare non aliquam nibh tristique. Aenean in ligula nisl. Nulla sed tellus ipsum. Donec vestibulum ligula non lorem vulputate fermentum accumsan neque mollis.\\~\\
    
    Sed diam enim, sagittis nec condimentum sit amet, ullamcorper sit amet libero. Aliquam vel dui orci, a porta odio. Nullam id suscipit ipsum. Aenean lobortis commodo sem, ut commodo leo gravida vitae. Pellentesque vehicula ante iaculis arcu pretium rutrum eget sit amet purus. Integer ornare nulla quis neque ultrices lobortis. Vestibulum ultrices tincidunt libero, quis commodo erat ullamcorper id.
\end{frame}
\begin{frame}
    \frametitle{A Model}
    Sed iaculis dapibus gravida. Morbi sed tortor erat, nec interdum arcu. Sed id lorem lectus. Quisque viverra augue id sem ornare non aliquam nibh tristique. Aenean in ligula nisl. Nulla sed tellus ipsum. Donec vestibulum ligula non lorem vulputate fermentum accumsan neque mollis.\\~\\
    
    Sed diam enim, sagittis nec condimentum sit amet, ullamcorper sit amet libero. Aliquam vel dui orci, a porta odio. Nullam id suscipit ipsum. Aenean lobortis commodo sem, ut commodo leo gravida vitae. Pellentesque vehicula ante iaculis arcu pretium rutrum eget sit amet purus. Integer ornare nulla quis neque ultrices lobortis. Vestibulum ultrices tincidunt libero, quis commodo erat ullamcorper id.
\end{frame}
\begin{frame}
    \frametitle{Connection to Standard Set Theory}
    Sed iaculis dapibus gravida. Morbi sed tortor erat, nec interdum arcu. Sed id lorem lectus. Quisque viverra augue id sem ornare non aliquam nibh tristique. Aenean in ligula nisl. Nulla sed tellus ipsum. Donec vestibulum ligula non lorem vulputate fermentum accumsan neque mollis.\\~\\
    
    Sed diam enim, sagittis nec condimentum sit amet, ullamcorper sit amet libero. Aliquam vel dui orci, a porta odio. Nullam id suscipit ipsum. Aenean lobortis commodo sem, ut commodo leo gravida vitae. Pellentesque vehicula ante iaculis arcu pretium rutrum eget sit amet purus. Integer ornare nulla quis neque ultrices lobortis. Vestibulum ultrices tincidunt libero, quis commodo erat ullamcorper id.
\end{frame}

\section{Height and Width Potentialism}
\begin{frame}
    \frametitle{Motivation}
    Sed iaculis dapibus gravida. Morbi sed tortor erat, nec interdum arcu. Sed id lorem lectus. Quisque viverra augue id sem ornare non aliquam nibh tristique. Aenean in ligula nisl. Nulla sed tellus ipsum. Donec vestibulum ligula non lorem vulputate fermentum accumsan neque mollis.\\~\\
    
    Sed diam enim, sagittis nec condimentum sit amet, ullamcorper sit amet libero. Aliquam vel dui orci, a porta odio. Nullam id suscipit ipsum. Aenean lobortis commodo sem, ut commodo leo gravida vitae. Pellentesque vehicula ante iaculis arcu pretium rutrum eget sit amet purus. Integer ornare nulla quis neque ultrices lobortis. Vestibulum ultrices tincidunt libero, quis commodo erat ullamcorper id.
\end{frame}
\begin{frame}
    \frametitle{Axioms}
    Sed iaculis dapibus gravida. Morbi sed tortor erat, nec interdum arcu. Sed id lorem lectus. Quisque viverra augue id sem ornare non aliquam nibh tristique. Aenean in ligula nisl. Nulla sed tellus ipsum. Donec vestibulum ligula non lorem vulputate fermentum accumsan neque mollis.\\~\\
    
    Sed diam enim, sagittis nec condimentum sit amet, ullamcorper sit amet libero. Aliquam vel dui orci, a porta odio. Nullam id suscipit ipsum. Aenean lobortis commodo sem, ut commodo leo gravida vitae. Pellentesque vehicula ante iaculis arcu pretium rutrum eget sit amet purus. Integer ornare nulla quis neque ultrices lobortis. Vestibulum ultrices tincidunt libero, quis commodo erat ullamcorper id.
\end{frame}
\begin{frame}
    \frametitle{Inconsistency?}
    Sed iaculis dapibus gravida. Morbi sed tortor erat, nec interdum arcu. Sed id lorem lectus. Quisque viverra augue id sem ornare non aliquam nibh tristique. Aenean in ligula nisl. Nulla sed tellus ipsum. Donec vestibulum ligula non lorem vulputate fermentum accumsan neque mollis.\\~\\
    
    Sed diam enim, sagittis nec condimentum sit amet, ullamcorper sit amet libero. Aliquam vel dui orci, a porta odio. Nullam id suscipit ipsum. Aenean lobortis commodo sem, ut commodo leo gravida vitae. Pellentesque vehicula ante iaculis arcu pretium rutrum eget sit amet purus. Integer ornare nulla quis neque ultrices lobortis. Vestibulum ultrices tincidunt libero, quis commodo erat ullamcorper id.
\end{frame}
\begin{frame}
    \frametitle{A Model}
    Sed iaculis dapibus gravida. Morbi sed tortor erat, nec interdum arcu. Sed id lorem lectus. Quisque viverra augue id sem ornare non aliquam nibh tristique. Aenean in ligula nisl. Nulla sed tellus ipsum. Donec vestibulum ligula non lorem vulputate fermentum accumsan neque mollis.\\~\\
    
    Sed diam enim, sagittis nec condimentum sit amet, ullamcorper sit amet libero. Aliquam vel dui orci, a porta odio. Nullam id suscipit ipsum. Aenean lobortis commodo sem, ut commodo leo gravida vitae. Pellentesque vehicula ante iaculis arcu pretium rutrum eget sit amet purus. Integer ornare nulla quis neque ultrices lobortis. Vestibulum ultrices tincidunt libero, quis commodo erat ullamcorper id.
\end{frame}
\begin{frame}
    \frametitle{Connection to Standard Set Theory}
    Sed iaculis dapibus gravida. Morbi sed tortor erat, nec interdum arcu. Sed id lorem lectus. Quisque viverra augue id sem ornare non aliquam nibh tristique. Aenean in ligula nisl. Nulla sed tellus ipsum. Donec vestibulum ligula non lorem vulputate fermentum accumsan neque mollis.\\~\\
    
    Sed diam enim, sagittis nec condimentum sit amet, ullamcorper sit amet libero. Aliquam vel dui orci, a porta odio. Nullam id suscipit ipsum. Aenean lobortis commodo sem, ut commodo leo gravida vitae. Pellentesque vehicula ante iaculis arcu pretium rutrum eget sit amet purus. Integer ornare nulla quis neque ultrices lobortis. Vestibulum ultrices tincidunt libero, quis commodo erat ullamcorper id.
\end{frame}

\begin{frame}
\frametitle{References}
\footnotesize{
\begin{thebibliography}{99}
    \bibitem[Smith, 2012]{p1} John Smith (2012)
    \newblock Title of the publication
    \newblock \emph{Journal Name} 12(3), 45 -- 678.
    \end{thebibliography}
}
\end{frame}

\begin{frame}
\Huge{\centerline{Thanks!}}
\end{frame}

\end{document} 